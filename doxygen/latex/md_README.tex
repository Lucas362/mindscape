\subsubsection*{Gênero}

Suspense.

\subsubsection*{Descrição}

Mind\+Scape é um jogo plataforma com elementos de terror e suspense. Como o jogo se situa na mente da protagonista, há monstros surreais que são criados com base em seus medos.

Além disso, o jogo trará a arte como elemento de história do jogo. Nas fases iniciais, haverá a presença de muitas cores que eventualmente vão desbotando e ficando preto e branco, para demonstrar a ciência da menina de sua própria situação.

\subsubsection*{História}

A história acompanha uma menina que encontra-\/se em coma por um acidente misterioso. Ela acorda em sua mente onde deverá enfrentar seus medos para que possa acordar antes que seja tarde demais.

\subsubsection*{Instruções}


\begin{DoxyItemize}
\item Andar para frente\+: →
\item Andar para trás\+: ←
\item Pular\+: ↑
\item Atacar\+: F
\item Pausar\+: Esq
\item Despausar\+: Barra de espaço
\end{DoxyItemize}

\subsubsection*{Envolvidos}

\tabulinesep=1mm
\begin{longtabu} spread 0pt [c]{*2{|X[-1]}|}
\hline
\rowcolor{\tableheadbgcolor}\PBS\centering {\bf Nome }&\PBS\centering {\bf Ocupação  }\\\cline{1-2}
\endfirsthead
\hline
\endfoot
\hline
\rowcolor{\tableheadbgcolor}\PBS\centering {\bf Nome }&\PBS\centering {\bf Ocupação  }\\\cline{1-2}
\endhead
\PBS\centering Luan Guimarães &\PBS\centering Programador \\\cline{1-2}
\PBS\centering Lucas Soares &\PBS\centering Programador \\\cline{1-2}
\PBS\centering Matheus Miranda &\PBS\centering Programador \\\cline{1-2}
\PBS\centering Victor Navarro &\PBS\centering Programador \\\cline{1-2}
\PBS\centering Bruna Figueirôa &\PBS\centering Artista \\\cline{1-2}
\PBS\centering Maria Monteiro &\PBS\centering Artista \\\cline{1-2}
\PBS\centering Amanda Santos &\PBS\centering Artista \\\cline{1-2}
\PBS\centering Natália Menezes &\PBS\centering Artista \\\cline{1-2}
\PBS\centering Aria Rita &\PBS\centering Música \\\cline{1-2}
\end{longtabu}
\subsubsection*{Dependências}

As dependências estão listadas pelo nome do pacote. Dessa forma, em distribuições {\itshape linux} baseadas em {\itshape Debian}, é necessário apenas rodar \char`\"{}sudo apt-\/get install nome\+Do\+Pacote\char`\"{}.
\begin{DoxyItemize}
\item libsdl2-\/dev
\item libsdl2-\/image-\/dev
\item libsdl2-\/ttf-\/dev
\item libsdl2-\/mixer-\/dev
\item make
\item gcc
\item g++
\item C\+Make (versão 3.\+1 ou maior)
\end{DoxyItemize}

\subsubsection*{Compilando e Executando}

Após clonar o repositório, entre na pasta Mind\+Scape. Nesse diretório, há um {\itshape script} sh responsável pela interação com o C\+Make e, portanto, com a compilação e execução. Dentro da pasta, os comandos e suas funções são\+:

\tabulinesep=1mm
\begin{longtabu} spread 0pt [c]{*2{|X[-1]}|}
\hline
\rowcolor{\tableheadbgcolor}\PBS\centering {\bf Comando }&\PBS\centering {\bf Efeito  }\\\cline{1-2}
\endfirsthead
\hline
\endfoot
\hline
\rowcolor{\tableheadbgcolor}\PBS\centering {\bf Comando }&\PBS\centering {\bf Efeito  }\\\cline{1-2}
\endhead
\PBS\centering {\ttfamily ./control.sh build} &\PBS\centering Gera os novos {\itshape Makefiles} \\\cline{1-2}
\PBS\centering {\ttfamily ./control.sh make} &\PBS\centering Usa os {\itshape Makefiles} gerados para compilar o jogo \\\cline{1-2}
\PBS\centering {\ttfamily ./control.sh run} &\PBS\centering Executa o jogo \\\cline{1-2}
\end{longtabu}
